\newcommand\newcitation[5]{
\newglossaryentry{#1}{
type=citation,
name={#2~(#3)}, 
description={#4},
symbol={#5}, 
plural={}}
}

%\newciation{label}{author}{yeard}{text}{ref_bib}
%\citeAtion{label}

\newcitation{ubeda2016logiciels}{
Ub{\'e}da, Stéphane et al.}{2016}{
L’existence d’une communauté active autour d’un code, qu’il s’agisse de développeurs ou d’utilisateurs, permet de disposer de ressources pour le faire évoluer, pour améliorer sa qualité et garantir une bonne réactivité face aux évolutions (scientifiques, techniques ou d’usages...).
}{
\cite{ubeda2016logiciels}}


\newcitation{papert1980mindstorms}{
Papert, Saymour}{1980}{
A programming language is like a natural, human language in that it favors certain metaphors, images, and ways of thinking. The language used strongly colors the computer culture. It would seem to follow that educators interested in using computers and sensitive to cultural influences would pay particular attention to the choice of language.
}{
\cite{papert1980mindstorms}}

\newcitation{kropotkine1904entraide}{
Kropotkine, Pierre}{1906}{
\large Dans la pratique de l’entraide, qui remonte jusqu’aux plus lointains débuts de l’évolution, nous trouvons ainsi la source positive et certaines de nos conceptions éthiques ; et nous pouvons affirmer que pour le progrès moral de l’homme, le grand facteur fut l’entraide, et non pas la lutte. Et de nos jours encore, c’est dans une plus large extension de l’entraide que nous voyons la meilleure garantie d’une plus haute évolution de notre espèce.
}{
\cite{kropotkine1904entraide}}

\newcitation{ISO}{
Organisation internationale de normalisation}{\textit{d{\'e}f.} Utilisabilité}{
L'utilisabilité correspond au degré auquel un système, un produit ou un service peut être utilisé par des utilisateurs spécifiés pour réaliser des objectifs spécifiés avec efficacité, efficience et satisfaction dans un contexte d’utilisation spécifié
}{
\citeURL{ISO}}

\newcitation{servo}{
Wikipédia, l'encyclopédie libre}{\textit{d{\'e}f.} Servomoteur}{
Un servomoteur (souvent abrégé en « servo », provenant du latin servus qui signifie « esclave ») est un moteur capable de maintenir une opposition à un effort statique et dont la position est vérifiée en continu et corrigée en fonction de la mesure. C'est donc un système asservi. Le servomoteur intègre dans un même boîtier, la mécanique (moteur et engrenage), et l’électronique, pour la commande et l'asservissement du moteur. La position est définie avec une limite de débattement d’angle de 180 degrés, mais également disponible en rotation continue.
}{
\citeURL{servo}}

\newcitation{robot-Oxford}{
Oxford Dictionary}{\textit{d{\'e}f.} Robot}{
A robot is a machine capable of carrying out a complex series of actions automatically, especially one programmable by a computer.
}{
\citeURL{robot-Oxford}}

\newcitation{tangible-Cambridge}{ 
Cambridge Dictionary}{\textit{d{\'e}f.} Tangible}{
Real and not imaginary; able to be shown, touched, or experienced
}{
\citeURL{tangible-Cambridge}}

\newcitation{robot-ife}{
Groupe de travail OCEAN, \gls{ife}}{\textit{d{\'e}f.} Robot}{
Un robot est un appareil mécatronique capable de manipuler des objets ou d'exécuter des opérations selon un programme fixe, modifiable ou adaptable. Alimenté en énergie, il est formé d'un microcontrôleur ainsi que d'un ou plusieurs capteurs et actionneurs.
}{
\citeURL{ife_robotique_2017}}

\newcitation{motivation}{
Darnon, Céline, pour Universalis}{\textit{d{\'e}f.} Motivation}{
La motivation peut être définie comme le processus psychologique responsable du déclenchement, du maintien, de l’entretien ou de la cessation d’une conduite. Elle est en quelque sorte la force qui pousse à agir et penser d’une manière ou d’une autre. Ainsi, le recours au concept de motivation s’avère particulièrement utile pour comprendre les cognitions et comportements dans bien des champs de l’activité humaine : l’éducation, le travail, la santé en sont quelques exemples.
}{
\citeURL{motivation}}

\newcitation{motivationCnrtl}{
\gls{cnrtl}}{\textit{d{\'e}f.} Motivation}{
\textit{\textbf{PSYCHOL}}. Ensemble des facteurs dynamiques qui orientent l'action d'un individu vers un but donné, qui déterminent sa conduite et provoquent chez lui un comportement donné ou modifient le schéma de son comportement présent.\\
~~~\textit{\textbf{PSYCHOPÉDAGOGIE}}. Ensemble des facteurs dynamiques qui suscitent chez un élève ou un groupe d'élèves le désir d'apprendre.
}{
\citeURL{motivationCnrtl}}

\newcitation{enseigner}{
Robine, Florence, \gls{DG}}{2013}{
"Enseigner est un métier qui s’apprend". Et j’ajouterais : "qui s’apprend tout au long de la vie" dans un processus subtil de développement progressif et intégré de savoirs, de savoir faire et de savoir être.
Pour autant, cette alchimie n’a rien de magique ni de garanti : c’est pourquoi il est indispensable de l’expliciter dans un référentiel de compétences professionnelles.
}{
\citeURL{BO-ref-prof}}

%"Nous n'avions aucun moyen supplémentaire, et pourtant nous avons les mêmes résultats. Les résultats tiennent vraiment aux leviers pédagogiques, et non pas au matériel utilisée ou à ma personnalité : ça tient à la posture de l'adulte, la capacité qu'il va avoir à soutenir le développement de ce que j'appelle les fonctions exécutives (les fondations biologiques de l'apprentissage)." alvarez

%« L’instructional design,c’est la partie avant les cours, quand on conçoit les tâches, les supports et les enseignements ». André Tricot


\newcitation{pseudonymisation}{
\gls{RGPD}, article 4}{\textit{d{\'e}f.} Pseudonymisation}{
on entend par pseudonymisation: le traitement de données à caractère personnel de telle façon que celles-ci ne puissent plus être attribuées à une personne concernée précise sans avoir recours à des informations supplémentaires, pour autant que ces informations supplémentaires soient conservées séparément et soumises à des mesures techniques et organisationnelles afin de garantir que les données à caractère personnel ne sont pas attribuées à une personne physique identifiée ou identifiable. 
}{
\citeURL{RGPD}}

\newcitation{methode}{
Wikipédia, l'encyclopédie libre}{\textit{d{\'e}f.} La méthodes expérimentales}{
Les méthodes expérimentales scientifiques consistent à tester la validité d'une hypothèse, en reproduisant un phénomène et en faisant varier un paramètre. Le paramètre que l'on fait varier est impliqué dans l'hypothèse, [\dots]~il s'agit de [le] modifier à l'aide d'un dispositif expérimental conçu pour permettre le contrôle de ces paramètres, dans le but de mesurer leurs effets et si possible de les modéliser.
}{
\citeURL{methode}}

\newcitation{num}{
Guillemot, Samuel}{2015}{
Une partie des objets autrefois tangibles, tels que les photos, les correspondances, les films et les musiques est aujourd’hui démocratisée sous forme numérique. Peut-on pour autant parler d’objets numériques ? Est-il possible de se les approprier ? Si oui, par quel(s) processus ? 
}{
\cite{guillemot2015objets}}

\newcitation{num2}{
Académie  des  sciences}{2013}{
L’impact considérable de l’informatique dans un nombre toujours croissant de domaines de l’industrie, de la communication, des loisirs, de la culture, de la santé, des sciences et de la société en général est universellement reconnu. On parle désormais d’un \og~monde numérique~\fg au sens large, qui s’appuie sur deux grands leviers, celui des matériels informatiques et celui de la science informatique.
}{
\citeURL{ac-rap}}

\newcitation{LVincent}{
Vincent, Luc, Enseignant ICN/ISN}{2016}{
J’avais déjà deux robots à disposition dans ma salle de cours et quand Inria nous a prêté 10 ErgoJr, nous avons constitué 10 groupes de 2 élèves. Cela me permet de travailler sur des thèmes différents et de créer des projets qui passionnent les élèves. Ces robots sont si fascinants que des élèves qui ne sont pas en 2\nd ICN viennent même assister aux cours en tant qu’invités ! J’ai pu aussi enrichir mes compétences en programmation avec Snap! et Python grâce à Poppy Education.
}{
\citeURL{prof-cite}}%Lycée des Graves (Gradignan)
\newcitation{JLCharles}{
Charles, Jean-Luc, maître de conférence, ENSAM}{2016}{
Cela fait deux ans que j’utilise Poppy Torso dans mes cours. Il permet de relever des défis d’enseignement pour donner envie aux élèves d’apprendre la conception, le design, la CAO, la mécanique, la programmation, les matériaux; toutes ces matières sont beaucoup plus difficiles à enseigner avec les anciennes méthodes. Les élèves deviennent acteurs et moteurs de leur cursus. Le fait qu’il y ait une plateforme matérielle à partager avec l’enseignant et les élèves, c’est un point de rencontre pédagogique, et ça, ça donne une force très importante à la pédagogie par projets.
}{
\citeURL{prof-cite}}%(Ecole nationale supérieur d'Arts et Métiers), Talence
\newcitation{GLassus}{
Lassus, Gilles, Enseignant Mathématique/ICN}{2016}{
Le robot ErgoJr est arrivé au lycée l’année dernière (2016). Je n’avais jamais enseigné la robotique auparavant et le fait de commencer le projet avec 8 robots (et les contraintes matérielles que cela puisse impliquer) était intimidant. Au final, le kit pédagogique a été très bien accueilli par les élèves en option ICN et le corps enseignant. Il est même devenu un élément incontournable de mon enseignement de l’informatique. De plus, Snap! a complètement changé mon approche de l’apprentissage de l’algorithmique.
}{
\citeURL{prof-cite}}%Lycée François Mauriac (Bordeaux)

\newcitation{alvarez}{
Alvarez, Céline}{interview FranceInter 2019}{
Nous n'avions aucun moyen supplémentaire, et pourtant nous avons les mêmes résultats. Les résultats tiennent vraiment aux leviers pédagogiques, et non pas au matériel utilisé ou à ma personnalité : ça tient à la posture de l'adulte, la capacité qu'il va avoir à soutenir le développement de ce que j'appelle les fonctions exécutives (les fondations biologiques de l'apprentissage).
}{
\citeURL{alvarez}}

\newcitation{mandela}{
Mandela, Nelson}{Discours de Boston, 1990}{
[~Students persevere,] because education is the most powerful weapon which we can use
}{
\citeURL{mandela}}
%at a speech, Madison Park High School, Boston, 23 June 1990 C'est une situation très troublante, car les jeunes d'aujourd'hui sont les leaders de demain", a-t-il déclaré aux étudiants. Il a exhorté les étudiants à "essayer autant que possible de rester à l'école"
