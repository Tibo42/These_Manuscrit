%Part2
\begin{concluPart}
Le robot et les ressources pédagogiques ont été développés en collaboration avec les utilisateurs et ont été évalués en situation réelle par des expérimentations (qui seront présentées en prochaine partie). En 2016, au début du projet Poppy Éducation, nous avons donc établi un groupe de travail d’enseignants volontaires pour co-créer des ressources pédagogiques en lien avec leurs besoins, leurs envies, et leurs contraintes. L'objectif étant d'avoir un outil simple d’appropriation, utilisable en salle de classe et accompagné de ressources \cro{clés en main} répondant aux objectifs des programmes scolaires. 
À ces fins, nous avions comme base de travail la plateforme robotique open-source Poppy, ainsi qu'un groupe d'enseignants issus des premières interactions ayant eu lieu sur le forum de la plate-forme susnommée.\par%
De ce fait, le Kit robotique ErgoJr a hérité d'un certain nombre de caractéristiques et notamment de l'aspect open-source sine qua non à la philosophie du projet. Et cela a été déterminant dans la construction du projet notamment via l'émergence d'une communauté. Cette communauté, bien que hétérogène (tant sur le plan des objectifs que des compétences initiales) a largement contribué à l'essor du projet: par des collaborations à la création de ressources ou de pratiques mais aussi par des productions propres, marquant une profonde appropriation de la plateforme.
Parmi ces productions, des robots mais aussi et surtout des activités et des projets pédagogiques (ou artistiques) visant à acculturer les individus à la robotique et plus généralement au numérique.\par%
Ce kit et ses développements \tiret{passés et futurs} doivent prendre en compte l’usage effectif qu’en font les enseignants en classe, leur propre appropriation des concepts informatiques et robotiques ainsi que leur capacité à réaliser une dissémination positive dans leur établissement.\par%
Aujourd'hui il existe des dizaines d'activités dont certaines sont référencées et mises en valeur sur le site web du projet.
On y retrouve aussi bien des activités d'initiation au robot que des activités portant sur des notions complexes, parfois spécifiques à des disciplines transversales. 
De plus, ces activités prennent différentes formes que ce soit par le langage de programmation utilisé ou par leur mise en œuvre pédagogique.\par%
Mais, quel est l'impact effectif de ces activités et projets, comment les évaluer, avec quels outils, quelles méthodes et pour quels objectifs? C'est ce que nous allons voir dans cette troisième partie.
\end{concluPart}
%\La combinaison des capteurs et des actionneurs permet de créer de l'interaction entre l'homme et la machine. Ainsi, selon les séquences pré-programmées dans le robot, les étudiants peuvent et doivent réaliser des codes plus ou moins poussés et exigeants pour mettre en œuvre des fonctionnalités intéressantes. Il est important de laisser de la liberté dans l'utilisation de ces capteurs et actionneurs pour permettre aux étudiants et professeurs de créer des fonctionnalités et de s'approprier le robot.
%Un kit robotique pédagogique destiné aux enseignants et leurs élèves en classe, se compose de différents éléments:~\\





