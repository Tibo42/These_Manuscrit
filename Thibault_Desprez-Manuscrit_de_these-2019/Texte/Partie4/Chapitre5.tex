\myChapter[orrélation connaissances~-~acceptabilité?]{C}{orrélation~~\break connaissances~-~acceptabilité?}

L'acceptabilité de la robotique et un concept mal défini et difficile à quantifier.
Ici, nous avons choisi de recourir au questionnaire \sht{EURO382} et \sht{NARS} pour mesurer celle-ci.\par\vspace{0.2cm}%
Ces mesures nous ont permis de mettre en évidence un lien entre les connaissances et l'acceptabilité.
Cependant ce lien n'est pas unique, ni unidirectionnel. En effet, nous avons relevé de nombreuses modalités d'exécution des activités robotiques en classe. Toutes ont eu un impact sur \sht{EURO382} mais aucune n'a été dominante. De plus, malgré des regroupements, aucun profil généré n'a permis de relier directement connaissances et acceptabilité. En revanche, nous pouvons remarquer que pratiquer des activités robotiques améliore d'un côté les connaissances en robotique et, d'un autre côté, l'acceptation de la robotique.\par\vspace{0.2cm}%
D'une manière plus générale, nous avons remarqué que certaines pratiques usuelles pouvaient en elles-mêmes impacter l'acceptabilité. Ici, cela se traduit par le fait que nommer son robot altère les résultats au \sht{NARS}.\par\vspace{0.2cm}%
Concernant la relation inverse, \cad qu'une plus grande acceptabilité favoriserait l'acquisition de connaissances; nos données quantitatives ne nous permettent pas de nous prononcer, et qualitativement, il n'a été noté aucun fait particulier allant dans ce sens.