\myChapter{L}{es choix de conceptions}

\subparagraph{Co-conception}
    La méthode employée durant la phase de conception, notamment pour les ressources pédagogiques, a été d'une très grande efficacité. Mais, concernant le développement du robot à proprement parler (hardware et software), cette conception avait un point de départ: la plateforme Poppy (et non une feuille blanche). De plus, autre particularité, la demande initiale venait de la population d'utilisateurs, en l'occurrence les enseignants ayant découvert cette plateforme qui par défaut était trop coûteuse et complexe pour être directement appliquée au monde pédagogique. Mais de ce fait, cette conception n'était pas neutre et, par exemple, l'intégration \tiret{dès les premières réunions} d'un public \cro{non-volontaire} aurait certainement orienté d'une autre manière, l'évolution du projet.
\subparagraph{Installation des softwares}
    Durant la phase de pré-conception, nous avons fait le choix d'avoir une architecture logicielle totalement embarquée dans le robot via une Raspberry~Pi.
    Ce choix avait été fait au vu des difficultés à installer de nouveaux logiciels sur un réseau des établissements scolaires. Cependant, nous constatons qu'il est au moins aussi difficile de connecter un nouveau périphérique telle qu'une Raspberry~Pi sur le réseau de l'établissement.% que d'installer de nouveaux logiciels.
    De plus, les gestionnaires de réseaux sont aujourd'hui de plus en plus habitués à devoir installer (sur la demande des enseignants) de nouveaux logiciels, or ce n'est pas le cas pour notre type de périphérique.
    L'évolution telle qu'a connu le Thymio: passant d'un logiciel principal à installer avec différents modules additionnels (à installer également), à un seul fichier exécutable contenant l'ensemble de la suite logicielle nécessaire, est une solution qui rétrospectivement semble plus adaptée.
\subparagraph{Connectivité}
    Un autre point critique est la connectivité du dispositif. À ce niveau, nous avions privilégié dans un premier temps le câble Ethernet (d'une fiabilité incomparable); cependant, son usage semble contraignant dans certains cas et les alternatives comme le wifi ou le bluetooth sont aujourd'hui encore sous-exploitées, malgré le développement logiciel qui a été effectué pour ces usages. Mais, il faut noter que les directives officielles en matière d'objets connectés dans les classes ne sont pas encore tout à fait claires.