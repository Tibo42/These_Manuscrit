\myChapter{L}{'impact motivationnel}

\paragraph{Les effets positifs}
    Qualitativement, nous observons un effet positif tant sur les élèves que sur les enseignants. D'une part, car nous constatons un rapprochement entre les représentations des individus et la réalité de la robotique et de l'informatique dans notre société aujourd'hui; mais aussi, car en intégrant ce type de dispositif, l'environnement scolaire se raccroche aux réalités sociétales. Ainsi, les croyances que l'individu porte sur lui et les tâches à réaliser sont plus en adéquation avec ses compétences réelles et les véritables enjeux de savoir manipuler les concepts et outils de la robotique.
    Quantitativement, les effets positifs \tiret{de certaines méthodes \etou formulations} se retrouvent dans les données que nous avons relevées.
\paragraph{Les effets négatifs}
    Ces dispositifs étant récents, les enseignants en charge de leur intégration n'ont pas reçu initialement de formation à leur manipulation. Cela engendre de réelles difficultés pour certains d'entre eux tant cette tâche d'auto-formation peut-être chronophage. De même, une fois formés, la mise en place de ces ressources requière généralement un investissement supérieur à des ressources classiques tel que le papier crayon. De plus, les bugs \tiret{inhérents à l'informatique} ajoutent une contrainte supplémentaire pour l'enseignant qui doit gérer, en plus des aspects humains et pédagogiques, des aspects techniques.
\paragraph{Corrélation motivation~-~connaissances}
    Les différents résultats que nous avons mis ici en évidence ne nous permettent pas aujourd'hui de dire si les kits robotiques pédagogiques sont des déterminants à la motivation scolaire, ni même s'il garantissent l'acquisition de connaissances en robotique ou dans des disciplines connexes comme l'informatique: tout dépend des usages effectués de ces outils par les enseignants. Plus généralement, il semble peu envisageable d'observer une causalité directe entre gain en motivation et gain en connaissances. Cependant, pour accroître ses connaissances il faut pratiquer, et donc s'engager dans l'activité, et parfois persévérer ou parfois décrocher puis se ré-engager; pour cela la motivation sera déterminante. Pour cela plusieurs modèles théoriques définissent des leviers permettant d'accroître la motivation; jouer sur ces leviers nous a permis de mettre en évidence certains impacts du kit.
    Cependant, il semble que le déterminant majeur de la motivation des élèves dans l'environnement scolaire ne soit pas les outils manipulés mais le rôle et la posture de l'enseignant.