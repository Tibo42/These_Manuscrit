\myChapter{L}{es choix d'évaluations}

La massification des données a induit la mise en place d'une nouvelle réglementation: la \sht{RGPD}. Entre autres pour garantir son respect, la \sht{CNIL} et autres comités d'éthique (\cf \sht{COERLE}) évaluent la nécessité réelle du recueil de données demandé dans le cadre d'une recherche. Ainsi, vouloir recueillir des données dans un seul but exploratoire augmente considérablement les délais de validation des expériences proposées à ces instances. Et, il en est de même pour les conditions de mise en place du protocole ou le type de question abordée.\par%
Avoir pu comparer des situations pédagogiques concrètes et classiques \tiret{telles que les conditions de construction d'un robot ou de présentation de concepts qui lui sont associés} a permis de fournir aux enseignants des clés de compréhension des outils qu'ils manipulent; mais nous aurions aimé aller plus loin en intégrant ces études dans un contexte plus long et écologique. Cependant, le recours à des questionnaires standardisés et la mise en place de sessions comparatives avec d'autres robots, dans différents contextes, a pu fournir des éléments concrets permettant d'évaluer ceux-ci. Notamment, nous avons vu que des facteurs parfois insignifiants, comme donner un nom à son robot, pouvaient impacter significativement notre perception de l'objet robotique, affectant potentiellement l'interaction; mais aussi, comment certains choix de conception pédagogique pouvaient être \cro{contre productifs} avec le cas de la notice de montage offrant un contrôle supérieur aux élèves mais diminuant leur perception de celui-ci. D'autres résultats \tiret{plus attendus} ont été confirmés, tel que l'avantage d'une manipulation tangible dans une situation d'apprentissage.
Mais, ici encore nous aurions aimé développer une étude plus proche du terrain; cependant, transposer directement la rigueur scientifique expérimentale à un environnement social écologique n'est pas possible. Et, malgré le respect des contraintes (notamment temporelles) qui avaient été identifiées durant la première année du projet, les enseignants n'ont pas été en mesure de respecter le calendrier expérimental proposé l'année suivante.
De nombreuses contraintes ponctuelles ou exceptionnelles sont venues les empêcher dans cette démarche. Et, on note que ces contraintes, à défaut d'être prévisibles, sont en revanche très fréquentes rendant par nature de telles études complexes à réaliser.\par%
Avoir pu conduire en parallèle deux types de protocoles: l'un longitudinal, étudiant l'utilisabilité du kit robotique pédagogique ErgoJr et l'acceptabilité de la robotique générale, et l'autre ponctuel, explorant les facteurs pouvant affecter l'auto-détermination des individus (leurs motivations), fut très enrichissant.
% aider l'enseignant à faire son choix de pratique (connu) dans ce domaine (inconnu)
% recueillir trop de donnée "principe de nécéssité démontré" cf CNIL