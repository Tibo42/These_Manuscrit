\myChapter[orrélation utilisabilité~-~appropriation?]{C}{orrélation~~\break utilisabilité~-~appropriation?}
%\paragraph{Les effets observés}\strut\par%%SUS ATT + etude de cas
    %\subparagraph{Non} 
    Un système, un produit, un outil, \etc utilisable de manière efficace, efficiente et satisfaisante, ne sera pas pour autant un objet accepté, même après une utilisation répétée. C'est face à un nouveau besoin \etou une nouvelle envie que l'individu va chercher à détourner des objets connus pour y répondre. Dans un souci d'efficacité, dans cette situation, un individu privilégiera des objets qu'il s'est déjà appropriés. Si aucun ne permet d'offrir une solution adéquate, il envisagera d'autres outils connus, qu'il s'appropriera pour les dériver et répondre à son problème. Mais, en parallèle, il peut également effectuer des recherches pour éventuellement découvrir de nouveaux outils qu'il n'a pas encore exploités. De ce constat factuel, nous pouvons présupposé qu'il n'existe pas de corrélation entre utilisabilité et appropriation.\par\vspace{0.2cm}%
    %\subparagraph{Mais}
    L'utilisabilité d'un outil va être déterminant dans le temps de prise en main et le degré de persévérance que va mettre un individu pour progresser, changeant la ligne de \sht{flow}.
    Même si, l'ensemble des enseignants de notre groupe de travail persévère dans l'utilisation du kit ErgoJr \tiret{qu'ils se sont tous appropriés (plus ou moins rapidement)} nous constatons malgré tout chez eux ce phénomène d'implication et d'engagement par intérêt, besoin ou envie propres.
    L'utilisabilité d'un outil n'est pas suffisante à son appropriation, mais elle semble en être un préalable, sans qu'il n'y ait de lien de corrélation; d'autres exemples de dispositifs, montrent que l'utilisabilité n'est même pas nécessaire à l'appropriation mais qu'elle découle de la nécessité de l'utilisation de l'outil.\par\vspace{0.2cm}%
    Offrir une diversité d'usages, de détournements, possédant des affordances fortes permettant une compréhension rapide des potentialités d'un dispositif sont des facteurs d'utilisabilité qui ont semblé ici jouer un rôle particulièrement important dans l'appropriation du kit ErgoJr par les enseignants.