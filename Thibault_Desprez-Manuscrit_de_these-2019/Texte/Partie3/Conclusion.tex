%Part3
\begin{concluPart}
Dès qu'il s'agit d'étudier l'humain, les moyens d'évaluation sont limités et les contraintes éthiques sont fortes.
S'offrent à nous des moyens issus de diverses disciplines telles que la sociologie avec les études de cas et les grilles d'observation ou la psychologie avec des protocoles, des métriques et des questionnaires spécifiques à des contextes expérimentaux particuliers. Ici nous avons choisi d'aborder notre analyse via, d'une part, une étude longitudinale et écologique et, d'autre part, via des études expérimentales ponctuelles. Suivant les cas, nous avons sélectionné les outils d'analyse nous semblant les plus pertinents.\par%
Ainsi, nous avons présenté une étude de cas montrant l'appropriation du Kit
par les enseignants partenaires du projet, qui pour certains avaient participé à sa conception. Quantitativement, nous avons montré grâce au \sht{SUS} et \sht{ATT} que le kit possédait une utilisabilité suffisante et une expérience utilisateur satisfaisante; préalable nécessaire à une appropriation à long terme.
De plus nous avons mis en évidence que l'utilisation de tels dispositifs avait un impact significatif sur l'acceptabilité de la robotique. Cette acceptabilité \tiret{ici mesurée par le questionnaire \sht{EURO382}} est en évolution pour la population générale, comparativement aux résultats obtenus en 2012. Mais surtout, ils diffèrent entre la population des élèves ayant pratiqué des activités avec ErgoJr et celle n'en ayant pas pratiqué. Cependant, malgré la définition de plusieurs modalités et le regroupement par profils, aucune interprétation générale concernant les spécificités des activités pratiquées n'a pu être extirpée. Mais nous constatons localement un certain nombre d'effets significatifs portés par ces modalités.\par%
En effet, il semble qu'à l'heure actuelle, il existe trop de facteurs inconnus impliqués pour pouvoir effectuer une analyse écologique pertinente d'un point de vue quantitatif.
Identifier ces facteurs est un objectif de notre approche expérimentale. Une de ses premières instanciations a été la recherche menée sur l'impact de la nomination d'un objet dans sa perception et son acceptation: donner un nom à un robot permet-il une plus grande proximité avec l'utilisateur?
Concernant le robot humanoïde Poppy, nous avons pu montrer que le simple fait de visionner des vidéos de ce robot mimant des émotions humaines améliorait les résultats obtenus au questionnaire \sht{NARS} (évaluant \cro{la peur} des robots).
D'autre part, nous avons montré que la passation de ce questionnaire en amont du visionnage des vidéos biaisait la reconnaissance des émotions mimées.
Sans ce billet, nous constatons que dans la version où le robot est appelé par son prénom \tiret{et non sa qualité} la reconnaissance est meilleure. Mais, de façon plus surprenante, nous avons aussi montré que la peur était maximale dans cette condition (mais toujours moindre que dans le cas de la passation seule du \sht{NARS}).\par%
Du côté de la motivation, comme nous l'avons dit, ce sont les processus mis en place par l'enseignant qui vont être le principal levier de motivation pour les élèves. Ainsi, dans le cadre d'une tâche de construction d'un robot, doit-il privilégier une approche modulaire ou une approche linéaire? Les résultats de notre étude montrent que, malgré l'avantage apparent, de la version modulaire sur le contrôle qu'ont les élèves sur la tâche qu'ils réalisent, c'est dans cette condition qu'ils perçoivent \tiret{lorsque relevé par le \sht{IMI}} un contrôle moindre. En revanche, comme attendu, cette condition permet l'exécution de la tâche de manière plus rapide: du fait de la simplicité à effectuer les sous tâches de façon parallèle.\par%
Outre la construction, un deuxième aspect inhérent à la robotique est, nous l'avons dit, la \bg{BSM}. Ainsi, nous nous sommes interrogés sur les préconisations à fournir aux enseignants pour maximiser son appréhension par les étudiants: est-il préférable d'expliquer à l'élève le principe de manière théorique ou, au contraire, est-il préférable de laisser l'élève explorer le comportement par pure manipulation? Sans réelle surprise, la manipulation  offre (sur cette session courte) de meilleurs résultats, tant sur la quantité d'efforts à fournir (moindre pour le \bg{GDemo}) que sur le \sht{QCM} passé en fin de session.\par%
Un dernier point que nous avons souhaité aborder concerne l'aspect tangible. Pour cela, nous avons comparé des activités similaires mais qui, d'un côté, exploitaient le robot ErgoJr dans une version simulée et d'un autre côté qui l'exploitait dans une version réelle. Les premiers résultats de cette étude pilote ont montré davantage de motivation à l'utilisation de la version réelle sans pour autant montrer de différence lors du \sht{QCM} post-activité. Cependant, une étude impliquant plus de sujets et des questionnaires mieux étalonnés ou standardisés est nécessaire avant de pouvoir tirer des conclusions fiables malgré la significativité de certains éléments.\par%
Plusieurs autres protocoles d'évaluation ont été développés afin de venir compléter les éléments ici mis en évidence ou pour explorer d'autres pistes de recherche.
La robotique pédagogique est une discipline nouvelle qui exploite certes des disciplines déjà connues et étudiées mais qui possède ses propres spécificités et caractéristiques qui potentiellement ont une influence sur les résultats déjà établis dans ces autres disciplines. Ainsi il convient, à minima, de reproduire ces résultats ou de définir de nouvelles orientations de recherches propres à cette nouvelle discipline comme le préconisait déjà Papert dans les années 70'. Dans cette optique nous discuterons dans la prochaine partie de différents points pouvant soutenir cette recherche.
\end{concluPart}